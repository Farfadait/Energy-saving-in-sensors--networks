\documentclass[a4paper]{report}
\usepackage[francais]{babel}
\usepackage[T1]{fontenc}
\usepackage[applemac]{inputenc}
\usepackage{geometry}
\usepackage{graphicx}
\usepackage[colorlinks=true]{hyperref}
\hypersetup{urlcolor=blue,linkcolor=black,colorlinks=true} 
\usepackage{algorithm,algorithmic}
\usepackage{listings}
\usepackage{amssymb}
\usepackage{setspace}
\usepackage{listings}
\usepackage{lscape}
\usepackage{mathabx}

\pagestyle{headings}
\thispagestyle{empty}
\geometry{a4paper,twoside,left=2.5cm,right=2.5cm,marginparwidth=1.2cm,marginparsep=3mm,top=2.5cm,bottom=2.5cm}
\begin{document}
\large
\setlength{\parskip}{5mm plus2mm minus2mm}
\lstset{language=C, showstringspaces=false, numbers=left, numberstyle=\tiny, tabsize=4}

Chlo� DESDOUITS \hfill M1 Informatique MOCA
\vfill
{\centering \Huge \bfseries Partie pratique du TP de complexit�, calculabilit� et algorithmique \par}
\vfill
15 d�cembre 2011 \hfill UM2

\tableofcontents
\thispagestyle{empty}
\pagenumbering{arabic}


\chapter{Introduction}\label{intro}
% qu'est ce qu'on a fait (plan)
% applications


\chapter{�tat de l'art}\label{etatArt}

\section{Notions pr�liminaires}

\section{Travaux relatifs}

\section{Synth�se}


\chapter{Analyse et r�flexion}\label{analyse}



\chapter{Simulations et r�sultats}\label{intro}



\chapter{Conclusion}\label{intro}


\vfill
{\raggedleft R�alis� avec \LaTeX{} \par}

\end{document}
